% If using the default install of texlive, you will need to install xurl through
%    sudo [apt|dnf|zypper] install texlive-xurl

\documentclass[12pt,a4paper]{article}
\usepackage[a4paper, total={6in, 8in}]{geometry}
\usepackage[utf8x]{inputenc}
\usepackage[T1]{fontenc}
\usepackage[english]{babel}
\usepackage[dvipsnames]{xcolor}

% document styling
\usepackage{helvet, titlesec}
\renewcommand{\familydefault}{\sfdefault}
\titlespacing\section{0pt}{12pt plus 4pt minus 2pt}{0pt plus 2pt minus 2pt}
\titlespacing\subsection{0pt}{12pt plus 4pt minus 2pt}{0pt plus 2pt minus 2pt}
\titlespacing\subsubsection{0pt}{12pt plus 4pt minus 2pt}{0pt plus 2pt minus 2pt}

\usepackage{
    tikz,
    verbatim, % multi line comments
    listings, % for inline code
    natbib, % references
    graphicx, % images
    % for maths
    mathptmx,
    amsmath,
    amssymb,
    longtable,
    % for \href{link}{text} and \url{link} hyperlinks
    hyperref,
    % I add the xurl package to allow URLs to be broken anywhere
    % otherwise, you will get overfull hbox errors
    xurl
}

\graphicspath{ {./images/} }

% custom symbols
\def\checkmark{\tikz\fill[scale=0.4](0,.35) -- (.25,0) -- (1,.7) -- (.25,.15) -- cycle;}
\newcommand\tab[1][1cm]{\hspace*{#1}}
\newcommand\mathspace[1][1cm]{\hspace*{4px}}

% use the \code{} expression to add inline code
\newcommand{\code}[1]{\colorbox{gray!30}{\texttt{#1}}}

\title{ 2 Instruction Set Architecture }
\author{ Hudson Newey }

\begin{document}

\begin{center}
    \huge{\textbf{ 2 Instruction Set Architecture }} \\
    \huge{\textbf{ Instruction Specification }}\\
    \hspace{200pt}\\
    % I usually like to keep this all lowercase to keep the character y-height the same
    \textsc{ reduced instruction set } \\
    \textsc{ version 1.0 }
    \vspace{280pt}\\
    % your name should be in all lowercase
    % this is because it is automatically converted to uppercase
    % causing character spacing to be incorrect if you capitalise your name
    \textsc{ hudson newey }
\end{center}

\newpage
\tableofcontents
\newpage

\section{Instruction Encoding}

When encoding instructions, the following format is used:

\begin{center}
\begin{tabular}{ c c c c }
    \textbf{31-26} & \textbf{25-21} & \textbf{20-16} & \textbf{15-0} \\
    \hline
    opcode & argument 1 & argument 2 & value \\
\end{tabular}
\end{center}

\subsection{Endianness}

All instructions use little endian format.

\subsection{Opcode Format}

\begin{itemize}
    \item $6$ bits are reserved for the opcode, meaning that we can represent up to $64$
different instructions.
    \item This allows us to expand upon the instruction set in the future.
    \item Most opcodes have both a value and register counterpart.
    \item Using the register version of an opcode will dereference the register and use
the value stored in the register.
    \item Alternatively, you can create pointers by using the registers location as a
value during the value version of the opcode.
\end{itemize}


\subsection{IO Register Format}

Registers are represented by 4 bits, allowing us to have up to $16$ registers.

\subsection{Value Format}

The value field is $16$ bits, allowing us to represent values from $0$ to
$65535$.

\section{Opcode Table}

\begin{center}
\begin{tabular}{ c c c c c }
    \textbf{Opcode} & \textbf{Binary} & \textbf{Instruction} & \textbf{Description} \\
    \hline
    x00 & b000000 & \texttt{NOOP} & Performs no operation \\

    \hline

    x01 & b000001 & \texttt{MOVE\_VAL} & Moves a value into a register \\
    x02 & b000010 & \texttt{MOVE\_REG} & Moves a registers value into a register \\

    \hline

    x03 & b000011 & \texttt{PUSH\_VAL} & Pushes a value onto the stack \\
    x04 & b000100 & \texttt{PUSH\_REG} & Pushes a registers value onto the stack \\

    x05 & b000101 & \texttt{POP\_VAL} & Pops a value from the stack \\
    x06 & b000110 & \texttt{POP\_REG} & Pops a value from the stack \\

    x07 & b000111 & \texttt{LOAD\_VAL} & Loads a value from memory \\
    x08 & b001000 & \texttt{NOOP} & reserved \\

    x09 & b001001 & \texttt{STORE\_VAL} & Stores a value in memory \\
    x0A & b001010 & \texttt{STORE\_REG} & Stores a registers value in memory \\

    \hline

    x0B & b001011 & \texttt{AND} & Performs a bitwise AND operation \\
    x0C & b001100 & \texttt{OR} & Performs a bitwise OR operation \\
    x0D & b001101 & \texttt{XOR} & Performs a bitwise XOR operation \\
    x0E & b001110 & \texttt{NOT} & Performs a bitwise NOT operation \\
    x0F & b001111 & \texttt{CMP} & Compares two values \\

    \hline

    x10 & b010000 & \texttt{ADD} & Adds two values together \\
    x11 & b010001 & \texttt{SUB} & Subtracts two values \\
    x12 & b010010 & \texttt{MUL} & Multiplies two values \\
    x13 & b010011 & \texttt{DIV} & Divides two values \\
    x14 & b010100 & \texttt{MOD} & Performs a modulo operation \\

    \hline

    x15 & b010101 & \texttt{JUMP} & Jump to memory location \\

    x16 & b010101 & \texttt{JUMP\_IF} & Jump to memory location if truthy \\

    \hline

    x3F & b111111 & \texttt{EXECUTE} & Execute the current command \\
\end{tabular}
\end{center}

\section{Register Table}

Each register is represented by a 4-bit value.

\begin{center}
\begin{tabular}{c c c}
    \textbf{Value} & \textbf{Binary} & \textbf{Register} \\
    \hline

    x0 & b0000 & \texttt{NULL} \\

    \hline

    x1 & b0001 & \texttt{CMD} \\
    x2 & b0010 & \texttt{RETURN} \\
    x3 & b0011 & \texttt{STACK} \\

    \hline

    x4 & b0100 & \texttt{ARG 1} \\
    x5 & b0101 & \texttt{ARG 2} \\
    x6 & b0110 & \texttt{ARG 3} \\
    x7 & b0111 & \texttt{ARG 4} \\
    x8 & b1000 & \texttt{ARG 5} \\
\end{tabular}
\end{center}

\end{document}

% If using the default install of texlive, you will need to install xurl through
%    sudo [apt|dnf|zypper] install texlive-xurl

\documentclass[12pt,a4paper]{article}
\usepackage[a4paper, total={6in, 8in}]{geometry}
\usepackage[utf8x]{inputenc}
\usepackage[T1]{fontenc}
\usepackage[english]{babel}
\usepackage[dvipsnames]{xcolor}

% document styling
\usepackage{helvet, titlesec}
\renewcommand{\familydefault}{\sfdefault}
\titlespacing\section{0pt}{12pt plus 4pt minus 2pt}{0pt plus 2pt minus 2pt}
\titlespacing\subsection{0pt}{12pt plus 4pt minus 2pt}{0pt plus 2pt minus 2pt}
\titlespacing\subsubsection{0pt}{12pt plus 4pt minus 2pt}{0pt plus 2pt minus 2pt}

\usepackage{
    tikz,
    verbatim, % multi line comments
    listings, % for inline code
    natbib, % references
    graphicx, % images
    % for maths
    mathptmx,
    amsmath,
    amssymb,
    longtable,
    % for \href{link}{text} and \url{link} hyperlinks
    hyperref,
    % I add the xurl package to allow URLs to be broken anywhere
    % otherwise, you will get overfull hbox errors
    xurl
}

\graphicspath{ {./images/} }

% custom symbols
\def\checkmark{\tikz\fill[scale=0.4](0,.35) -- (.25,0) -- (1,.7) -- (.25,.15) -- cycle;}
\newcommand\tab[1][1cm]{\hspace*{#1}}
\newcommand\mathspace[1][1cm]{\hspace*{4px}}

% use the \code{} expression to add inline code
\newcommand{\code}[1]{\colorbox{gray!30}{\texttt{#1}}}

\title{ 2 Instruction Set Architecture }
\author{ Hudson Newey }

\begin{document}

\begin{center}
    \huge{\textbf{ 2 Instruction Set Architecture }} \\
    \huge{\textbf{ Instruction Specification }}\\
    \hspace{200pt}\\
    % I usually like to keep this all lowercase to keep the character y-height the same
    \textsc{ reduced instruction set } \\
    \textsc{ version 1.0 }
    \vspace{280pt}\\
    % your name should be in all lowercase
    % this is because it is automatically converted to uppercase
    % causing character spacing to be incorrect if you capitalise your name
    \textsc{ hudson newey }
\end{center}

\newpage
\tableofcontents
\newpage

\section{Instruction Encoding}

When encoding instructions, the following format is used:

\begin{center}
\begin{tabular}{ c c c c }
    \textbf{31-26} & \textbf{25-21} & \textbf{20-16} & \textbf{15-0} \\
    \hline
    opcode & output register & input register & value \\
\end{tabular}
\end{center}

\subsection{Opcode Format}

$6$ bits are reserved for the opcode, meaning that we can represent up to $64$
different instructions.

This allows us to expand upon the instruction set in the future.

\subsection{IO Register Format}

Registers are represented by 4 bits, allowing us to have up to $16$ registers.

\subsection{Value Format}

The value field is $16$ bits, allowing us to represent values from $0$ to
$65535$.

\end{document}
